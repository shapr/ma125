% -*- fill-column: 110 -*-
\documentclass[12pt]{article}
\usepackage[margin=1in]{geometry}
\usepackage{hyperref}
\usepackage{amsmath}
\usepackage{amsthm}
\begin{document}
\section*{Calculus}
\section{Review}

\subsection{Factoring}
First we review basic factoring.

\begin{math}
a^2-b^2 = (a+b)(a-b)\\
a^3-b^3 = (a-b)(a^2+ab+b^2)\\
a^3+b^3 = (a+b)(a^2-ab+b^2)\\
a^2+2ab+b^2 = (a+b)^2\\
a^2-2ab+b^2 = (a-b)^2
\end{math}

\subsection{Radicals}

$\sqrt[n]{x} = x^{\frac{1}{n}}$

So what would be the result of $ \sqrt[n]{\frac{x}{y}} $ ?

more radicals

$x^{-\frac{1}{n}} = \dfrac{1}{x^{\frac{1}{n}}} = \dfrac{1}{\sqrt[n]{x}}$
\section{2013-01-14 Mon limit of a function and limit laws}
start with 2.2 and will pick up 2.1 later

consider $f(x) = \frac{x^3-x^2}{2x-2}$ near c = 1

we want to know the behavior of function f near c

\begin{flalign}
x &= f(x)\\
0.9 &= 0.405\\
0.99 &= 0.49005\\
0.9999 &= 0.499900005\\
1.0001 &= 0.500100005\\
1.01 &= 0.51005\\
1.1 &= 0.605\\
\end{flalign}
it appears that $f(x) = \frac{x^3-x^2}{2x-2}$ near c = 1 gets closer
to 0.5 as x gets closer to 1.

Notice that $f(x) = \frac{x^3-x^2}{2x-2} = \frac{x^2(x-1)}{2(x-1)} =
\frac{x^2}{2}$ if x is $\neq$ to 1
and f(x) is undefined if x = 1

(insert parabola graph of this equation)

we can see from the graph that as x gets closer to 1, f(x) gets closer
to $\frac{1}{2}$.

We say that the limit of the function $\frac{x^3-x^2}{2x-2}$ as
x approaches 1 is equal to $\frac{1}{2}$.

$\lim_{x \to 1} \frac{x^3-x^2}{2x-2} = \frac{1}{2}$ (how do I get the
lim stuff COMPLETELY under the lim word?)

\subsection{Informal definition of a limit}
suppose f(x) is defined on an open interval about c, except possibly
at c itself. If f(x) is arbitrarily close to L (as close to L as we
like) for all x sufficiently close to c, we say that f approaches the
limit L as x approaches c, and write $\lim_{x \to c}, f(x) = L $

NOTE it is important to note that $\lim_{x \to c} f(x)$
DOES NOT depend on the value of f at c.
Instead what matters is the value of f NEAR c.

Now we graph a function f

funny triangle I don't quite understand
at zero it's 7, at 2 it's 3 at 3.5 it's roughly zero
$\lim_{x \to 2} f(x) = 3$

now we put a hole at x = 2, showing that x is not defined there.
so we got closer and closer to 3, but never quite reach it.

We draw the graph again, but we add a point 2,6 such that the y value AT 2 is definitely 6. Still doesn't
change the result of the limit because what matters is NEAR c, not AT c.

Another graph,
$f(x) = \left\{ \begin{array}{l}
1, x < 0\\
2, x \geq 0 \end{array} \right. $
What's the limit of f(x) as it approaches 0? The limit does not exist!

but what about $f(x) = \frac{1}{x^2}$ and we want $\lim_{x \to 0} f(x)$.
We have roughly a solar chimney graph, where y asymptotically approaches 0

Is there a single finite number that our y-value is approaching as our x gets closer to zero? NO!
because this increases without bound

next graph is: $f(x) = \left\{\begin{array}{l}
0,x \leq 0\\
sin \frac{1}{x}, x > 0
\end{array} \right. $
limit does not exist because the value is oscillating

\subsection{Limit Laws}
If L,M, c and k are real numbers and
$\lim_{x \to c} f(x) = L$ and $\lim_{x \to c} g(x) = M$, then
$\lim_{x \to c}(f(x) + g(x)) = L+M$
that is, the sum of the limits is the limit of the sums?
$\lim_{x \to c}(f(x) - g(x)) = L-M$
that requires the difference to be the same
$\lim_{x \to c}( k \times f(x)) = k \times L$

\section{2013-01-15 More Limit Laws}

$\lim_{x \to c}(f(x) \times g(x)) = \lim_{x \to c}f(x) \times \lim_{x \to c}g(x) = L \times M$

$\lim_{x \to c} \frac{f(x)}{g(x)} = \lim_{x \to c} \frac{\lim_{x \to c} f(x)}{\lim_{x \to c} g(x)} \lim_{x \to
  c}g(x) \neq 0 = \frac{L}{M},M \neq 0$

$\lim_{x \to c}[f(x)]^n = [\lim_{x \to c}f(x)]^n = L^n$,n is a positive integer
the previous line expands to $\lim_{x \to c}[f(x) \times f(x) ... f(x)]$ for each n in the exponent.
and is the same as $=L \times L ... L = L^n$

$\lim_{x \to c}\sqrt[n]{f(x)} = \sqrt[n]{\lim_{x\to c}f(x)} = \sqrt[n]{L}$
If n is even, we assume $\lim_{x \to c}f(x) = L > 0$

NB. Our instructor suspects that the above law is actually $L \geq 0$ because you can take the nth root of 0.

\subsection{Basic Limits}

$\lim_{x \to c}x = C$
if we graphed this as y = x, it's just a line with slope of positive 1
as x gets closer to C, x gets closer to C. Obvious.

$\lim_{x \to c}k = k$ as x approaches k, x approaches k!

$$
\lim_{x \to 8} x = 8\\
\lim_{y \to 23} y = 23\\
\lim_{h \to 0} 7 = 7
$$

Find the following limits
$\lim_{x \to 5}2x^2 - 10\\
=\lim_{x \to 5}2x^2 - \lim_{x \to 5}10\\
=2\lim_{x \to 5}x^2 - \lim_{x \to 5}10\\
=2(\lim_{x \to 5}x)^2 - \lim_{x \to 5}10\\
=2(5)^2 - 10\\
=2(25) - 10\\
= 40$

$\lim_{x \to 4}\frac{3x+1}{x+2}\\
=\frac{\lim_{x \to 4}3x+1}{\lim_{x \to 4}x+2}\\
=\frac{3\lim_{x \to 4}x+\lim_{x \to 4}1}{\lim_{x \to 4}x+\lim_{x \to 4}2}\\
=\frac{3(4)+1}{4+2}
$

\subsection{Limits of Polynomials}
If a polynomial $P(x) = a_n x^n + a_{n-1}x^{n-1}+ ... + a_1x + a_0$, then $\lim_{x \to c} P(x)= P(c)$

Instructor points out that all the above steps ended up substituting the limit value into the polynomial.

\subsection{Limits of Rational Functions}
If P(x) and Q(x) are polynomials and $Q(c) \neq 0$, then $\lim_{x \to c}\frac{P(x)}{Q(x)} = \frac{P(c)}{Q(c)}$

now for some examples

$\lim_{x \to -1}(2x^3 - 3x + 1)\\
= (2(-1)^3 - 3(-1) + 1)\\
= (-2 + 3 + 1)\\
= 2$

Can we plug in zero for this one? Yes! Because the denominator will not end up being zero.
$\lim_{x \to 0}\frac{5x^2 + x - 1}{2x + 3}\\
=\frac{5(0)^2 + 0 - 1}{2(0) + 3}
$

This one will have a zero in the denominator.
$\lim_{x \to 2}\frac{x^2 +2x-8}{x-2}\\
=\lim_{x \to 2}\frac{(x-2)(x+4)}{x-2} = x+4 \mathrm{if}  x \neq 2$
since we are only interested in what happens when x is NEAR 2 but not equal to 2, we can use the expression
x+4 for our function when taking the limit as k approaches 2.

Now to actually work the problem:

$
\lim_{x \to 2}\frac{x^2 +2x-8}{x-2} = \lim_{x \to 2}\frac{(x-2)(x+4)}{x-2} = \lim_{x \to 2}(x+4) = 2+4 = 6
$

NB. Don't drop your limit too early. Even though your homework is online, don't just do it in your head. Tests
will be done on paper, so don't just give an answer.


$\lim_{x \to 25}\frac{\sqrt{x}-5}{x - 25}$ can't just plug in 25\\
$\lim_{x \to 25}\frac{\sqrt{x}-5}{x - 25} \times \frac{\sqrt{x + 5}}{\sqrt{x + 5}}$
$\lim_{x \to 25}\frac{x-25}{(x - 25)(\sqrt{x}+5}$
$\lim_{x \to 25}\frac{1}{\sqrt{x}+5}$

$\lim_{x \to -3}\frac{x+3}{\frac{1}{x}+\frac{1}{3}}$
$\lim_{x \to -3}\dfrac{x+3}{\frac{1}{x}+\frac{1}{3}} \times \dfrac{3x}{3x}$
$\lim_{x \to -3} \dfrac{(x+3)(3x)}{3+x}$
$\lim_{x \to -3} 3x$
$3(-3) = -9$

\subsection{Squeeze Theorem}

Suppose that $g(x) \leq f(x) \leq h(x) \forall x$ in some open interval\footnote{an open interval necessarily
  means that one or more end points are open (not included), that is would be an open circle in a number line} containing c, except possibly at $x=c$
itself.

$\lim{_{x \to c}}g(x) = \lim{_{x \to c}}h(x) = L$. Then  $\lim_{x \to c}f(x) = L$.
This is true because f is squeezed between h and g.
$\frac{1}{2}$

\section{2013-01-16 2.2 continued, squeeze theorem}

if $0 \leq f(x) \leq c$ for some real number c, then prove
function f is bounded between the number 0 and c
we want to prove that $\lim_{x \to 0} x^2f(x) = 0$

(to use the squeeze theorem, we need the limits of the two outer functions)

Proof
suppose that  $0 \leq f(x) \leq c$ for some real number c then
$0 \leq x^2f(x) \leq cx^2$ Also $\lim_{x \to 0}0 = 0$ and $\lim_{x \to 0}cx^2 = 0$. Thus by the squeeze
theorem, $\lim_{x \to 0}x^2f(x) = 0$.

Prove that $\lim_{x \to 0}x^4 \cos \frac{1}{x} = 0$.

We'll want to get to $\leq x^4\cos \frac{1}{x} \leq$, that is, we need the limits of the two outer functions.

We know that $-1 \leq \cos t \leq 1$ and $-1 \leq \cos \frac{1}{x} \leq 1 \forall x \neq 0$

We distribute $x^4$ so we get $-x^4 \leq x^4 \cos \frac{1}{x} \leq x^4$

\begin{proof}
Since $-1 \leq \cos t \leq 1$ for every real number t then $-1 \leq \cos \frac{1}{x} \leq 1$ for every $x \neq
0$. Thus $-x^4 \leq x^4 \cos \frac{1}{x} \forall x \neq 0$. Also $\lim_{x \to 0} -x^4 = 0$ and $\lim_{x \to 0}
x^4 = 0$. Hence, by the Squeeze Theorem, $\lim_{x \to 0} x^4 \cos \frac{1}{x} = 0.$
\end{proof}
Would be the same for $\sin$ instead of $\cos$, but would not be bounded for every x out there?

find

This should help you out with some of the homework.
$\lim_{x \to \frac{\pi}{2}} \sin x = sin \frac{\pi}{2} = 1$

\subsection{2.3 The precise definition of a limit}
If we can make f(x) arbitrarily close to L as long as we can make f(x) close to c. What?

Now we get a more formal definition of close enough.

We have agreed that $\lim_{x \to c}f(x) = L$ if we can cause f(x) to be arbitrarily close to L (as close as we
like) by taking x sufficiently close (enough) to c.
So there is a positive number $\delta$ such that x is in $(c-\delta, c+\delta)$
$c- \delta < x < c + \delta$
$-\delta < x -c < \delta$
$|x - c| < \delta$

The above means x is close to c.


Let's say that f(x) is close to L if f(x) lies within an open interval centered at L. So there is a positive
number $\epsilon$ such that f(x) is in $(L-\epsilon{},L+\epsilon{})$
$
L-\epsilon < f(x) < L + \epsilon
$
$-\epsilon < (x) -L < \epsilon$
$|f(x)-L| < \epsilon$

The above means f(x) is close to L.

We will now use these together.

\subsection{Formal Definition of a Limit}
Let f be a function defined on some open interval\footnote{a closed interval means that the range of the
  function is \underline{continuous} over the real numbers.} containing c, except possibly at c itself.

Then the$\lim_{x \to c} f(x) = L$ if for every $\epsilon > 0 \exists \delta > 0$ such that if $0 < |x-c| <
\delta{}$\footnote{This is where x is sufficiently close to c, note that x cannot equal c, thus we have the
  greater than zero constraint} then we will have our $|f(x)-L| < \epsilon $\footnote{f(x) is arbitrarily close to L}.

Epsilon is how far away from the limit you are, delta is how far away from the ...


$\lim_{x \to c}f(x) = L$ we have a sine graph of f, where we examine the maximum vertical interval we can have
that will satisfy $(c- \delta,c + \delta)$

Prove $\lim_{x \to 3}(2x +4) = 10$

\begin{proof}
Given $\epsilon > 0$. We seek a $\delta > 0$ such that if $0 < |x-c| < \delta$ then $|f(x)-L|< \epsilon$.

we substitute 3 for c and 10 for L. (why?)

if $0 < |x-3| < \delta$ then $|(2x-4)-10|< \epsilon$
if $0 < |x-3| < \delta = ?$ then $|(2x+4)-10| = |2x-6| = |2(x-3)| = 2|x-3| $
we know that $ 2|x-3| < 2 \delta = 2(?) = 2(\frac{\epsilon}{2} = \epsilon)$ what?
\end{proof}

\section{2013 01 13}
\subsection{left and right limits}
Theorem a function f has a limit as x approaches c iff and only if it has a left-hand and right-hand limit
there and these one-sided limits are equal

$\lim_{x \to c}f(x) = L \iff \lim_{x \to c -} = L = \lim_{x \to c +}f(x)$

ex: given the graph of f, find:

$\lim_{x \to -2^-}f(x)$ = 1
$\lim_{x \to -2^+}f(x)$ = 4
$\lim_{x \to -2}f(x)$ does not exist because limit from the left not equal to the limit from the right
$\lim_{x \to 1^-}f(x)$ = 1
$\lim_{x \to 1^+}f(x)$ also 1
even though there's a hole at one, it's about what it approaches, not what it equals.
$\lim_{x \to 1}f(x)$ = 1
$\lim_{x \to 2^-}f(x)$ = 4 even though
$\lim_{x \to 2^+}f(x)$ does not exist, not defined to the right of past x=2
$\lim_{x \to 2}f(x)$

the + means approach from the right,
the - means approach from the left

graph is -2 to 2
comes in from the left to 2, but open circle at 2
is parabola from -2 to 2
with open circle at 1(y=4) and 2 (y=4), at 2 it is 3

\subsection{limits involving $\frac{\sin{\theta{}}}{\theta{}}$}

theorem $\lim_{\theta \to 0} \frac{\sin \theta}{\theta} = 1$ ($\theta$ in radians)

also $\lim_{\theta \to 0} \frac{\theta}{\sin \theta} = 1$

we will use this known limit to help us find other limits

ex find the limit

$\lim_{x \to 0} \dfrac{x}{\sin 3x} = 1$

$\lim_{x \to 0} \dfrac{3x}{3\sin 3x} = 1$

$\dfrac{1}{3} \lim_{x \to 0} \dfrac{3x}{\sin 3x} = 1$

$\dfrac{1}{3}(1)$

$\dfrac{1}{3}$

wait WHAT?

$\lim_{x \to 0}\dfrac{x^2 - x + \sin x}{2x}$

$\lim_{x \to 0}\dfrac{x^2 - x}{2x} + \dfrac{\sin x}{2x}$

limit of a sum is the sum of the limits

$\lim_{x \to 0}\dfrac{x^2 - x}{2x} + \lim_{x \to 0} \dfrac{\sin x}{2x}$

can't go to zero, so we factor the numerator

$\lim_{x \to 0}\dfrac{x(x - 1)}{2x} +\dfrac{1}{2} \lim_{x \to 0} \dfrac{\sin x}{x}$

cancel on the left, I think the right side becomes 1 because of this section's theorem

$\lim_{x \to 0}\dfrac{x - 1}{2} +\dfrac{1}{2} (1)$

$\dfrac{0-1}{2} + \dfrac{1}{2}$

ends up being zero

$\lim_{x \to 0} 6x^2 \cot x \csc 2x$

$\lim_{x \to 0} 6x^2 \dfrac{\cos x}{\sin x} \dfrac{1}{\sin 2x}$

supposedly the term above is exactly the same as the term below, but I don't immediately see it

$\lim_{x \to 0} 3\cos x \dfrac{x}{\sin x} \dfrac{2x}{\sin 2x}$

$(\lim_{x \to 0} 3\cos x)(\lim_{x \to 0} \dfrac{x}{\sin x})(\lim_{x \to 0} \dfrac{2x}{\sin 2x})$

$(\lim_{x \to 0} 3\cos x)(1)(1)$

$3 \cos 0$

$3(1)$

3

finish of 2.4

\subsection{2.5 continuity}

Intuitively, a function f is continuous at c if you can draw the graph of f without picking up your pen at c.

Continuous at c is a line that goes straight through c.

Discontinuous at c is an open circle at c.

There are multiple types of discontinuities.

Jump Discontinuity is where a value jumps from one value to another at a point, and then continues along that
new line.

Removable Discontinuities function has a hole at c (but is continuous otherwise).
Or where point c has a value that is not on the line graphed.

Infinite discontinuity is an asymptote at c.

Oscillating discontinuity $y = \sin\dfrac{1}{x}$

A function f is continuous at c if the following conditions are satisfied:
f(c) is defined
$\lim_{x \to c}f(x)$ exists
$\lim_{x \to c}f(x) = f(c)$ that is, the limit at the point must be the same as the value of the function at
that point

A function f is continuous on an open interval (a,b) if it is continuous for every number in (a,b)

A function f is continuous on the closed interval [a,b] if it is continuous on (a,b),$\lim_{x \to a^+}f(x) =
f(a)$ and $\lim_{x \to b^-}f(x) = f(b)$

second condition means f is continuous from the right at a, and third condition means f is continuous from the
left at b

Classify the discontinuities of f

$f(x) = \left\{\begin{array}{l}
x^2-1,x < 1\\
4-x, x \geq 1
\end{array} \right. $

Will be a parabola on the left side of 1, and a line with a slope of -1 to the right of x=1.

It's a jump discontinuity.


More of 2.5 tomorrow, 2.4 homework is due Thursday!



\section{2013-01-23 2.5 continuity}

\subsection{properties of continuous functions}

if the function f and g are continue at c, then the following are continuous at c:
\begin{itemize}
\item sums f+g
\item differences f-g
\item constant multiples
\item products f * g
\item quotients $\dfrac{f}{g}$ provided $g(c) \neq 0$
\item powers $f^n$ where n is a positive integer
\item roots $\sqrt[n]{f}$, provided root is defined on an open interval containing c, and n is a positive integer
\end{itemize}

\begin{proof}
Proof of products:

Let f and g be functions which are continuous at c.
Then, by the definition of continuity $\lim_{x \to c}f(x)=f(c)$ and $\lim_{x \to c}g(x)=g(c)$.
so $\lim_{x \to c} f(x) \times g(x)$.
 $\lim_{x \to c} f(x) \lim_{x \to c} g(x)$.
 $f(c) g(c)$.
$(f . g) (c)$
thus f.g is continuous at c.

\end{proof}


Find and classify the discontinuities of f if $f(x) = \dfrac{x-1}{x^2+x-2} $ at which points is f continuous?

For denominator, only have to worry about zero values, which would make it discontinuous.

$x+x-2=0$
$(x-1)(x+2)=0$
$x-1=0$ or $x+2=0$
$x=1$ and $x=-2$

factoring the denominator means the original function ends up looking like $f(x) = \dfrac{x-1}{(x-1)(x+2)} = \dfrac{1}{x+2} $
if $x \neq 1$ (if x = 1, denominator becomes zero, which is illegal!)

At -2 the graph shows an infinite discontinuity, and at +1 it shows a removable discontinuity.

Where is f continuous? $(-\infty,-2)\cup(-2,1)\cup(1,\infty)$

clearly f(x)=x and g(x) = k where k is a constant are continuous on $(-\infty,\infty)$.

$\lim_{x \to c}f(x) = \lim_{x \to c}x = c = f(c)$
$\lim_{x \to c}g(x) = \lim_{x \to c}k = c = g(c)$

fact: every polynomial function is continuous on $(-\infty,\infty)$
fact: rational functions and trigonometric functions are continuous at every number in their domain
fact: the inverse function of any function continuous on an interval is continuous over its domain.

theorem: if f is continuous at c and g is continuous at f(c), then the composite function $g \cdot f$ is
continuous at c.

$\lim_{x \to c}(g \cdot f)(x) = \lim_{x \to c}g(f(x)) = \lim_{x \to c}g(\lim_{x \to c}f(x)) = \lim_{x \to
  c}g(f(c)) = g(f(c)) = (g\cdot f)(c)$

more generally

theorem:

if g is continuous at b and $\lim_{x \to c}f(x) = b$, then $\lim_{x \to c}g(f(x)) = g(\lim_{x \to c}f(x)) =
g(b)$

find $\lim_{x \to \pi}\sin(x - \sin x)$

because $\sin$ is continuous on all real numbers, we can bring the limit inside:

$\sin(\lim_{x \to \pi}(x - \sin x))$

$\sin(\pi - \sin \pi)$

$\sin \pi$

0

\subsection{Intermediate value theorem}
if f is a continuous function on a closed interval $[a,b]$ and if y is any value between f(a) and f(b), then $y
= f(c)$ for some c in $[a,b]$.

(The y above has an open circle after it at the height of the base of the letter?, it's called Y )

Graphically

\section{2013-01-24 intermediate value theorem}

if f is a continuous function on a closed interval [a,b], and if y nought is any value between f(a) and f(b),
then $y_0$ = f(c) for some c in [a,b].

prove that the equation $x^3 = -1 + 4x$ has a solution in [1,2].

$x^3 = -1 + 4x$

$x^3 - 4x = -1$

we want to show that $c^3 - 4c = -1$ for some c in [1,2] ( $f(c) = -1$ for some c in [1,2]).

We need to show that f is continuous on [1,2] and show that -1 is between f(1) and f(2) THEN we have this conclusion.

\begin{proof}
Let $f(x) = x^3 - 4x$ since f is a polynomial function, then it is continuous for all real numbers. Thus f is
continuous on [1,2]. Also, $f(1) = -3$ and $f(2) = 0$. So $f(1) < -1 < f(2)$. Hence by the intermediate value
theorem, $f(c) = -1$ for some c in [1,2]. So $c^3-4c=-1$ or equivalently $c^3=-1+4c$ for some c in
[1,2]. Therefore, $x^3 = -1 + 4x$ has a solution in [1,2].
\end{proof}
This could be done differently:
$0 = -1 + 4x - x^3$
$f(x) = -1 + 4x - x^3$
we would want to show $f(c) = 0$ for some c in [1,2].


Ex: prove that the function $f(x) = x^5 - x^2 - 4$ has at least one real zero.

We want $f(x) = 0 \leftarrow y_0$. 

We have to figure out our interval.

$f(0) = 0^5 - 0^2 - 4 = -4$ <0

$f(1) = 1^5 - 1^2 - 4 = -4$ <0

$f(2) = 2^5 - 2^2 - 4 = 24$ >0

The interval from (1,2) would work because one is less than zero, and one is greater than zero.

so $[a,b] = [1,2]$ and $y_0 = 0$ with $f(x) = x^5 - x^2 - 4$.


\begin{proof}
Let $f(x) = x^5 - x^2 - 4$. Since f is a polynomial function, then it is continuous for all real
numbers\footnote{Every polynomial function is continuous for all real numbers.} Thus f is continuous on the
closed interval [1,2]. Also $f(1) = -4$ and $f(2) = 24$. So $f(1) < 0 < f(2)$. Hence, by the intermediate
value theorem, $f(c) = 0$ for some c in [1,2]. Therefore, $f(x) = x^4 - x^2-4$ has a zero in [1,2].
\end{proof}

\subsection{2.6 Limits involving Infinity}
\begin{itemize}
\item Large negative number - a negative number whose absolute value is large.
\item $\infty$ is a not a number. It is a symbol.
\item We want to investigate the behavior of a function when x becomes increasingly large ($x \to \infty$) or when
  x becomes increasingly large negative ($x \to -\infty$).
\end{itemize}

Informal definition

We say $\lim_{x\to\infty} f(x) = L$ if the values of f(x) can be made arbitrarily close to L by taking x
sufficiently large.

We say $\lim_{x\to\infty}f(x)=L$ if the values of f(x) can be made arbitrarily close to L by taking x
sufficiently large negative.

Ex: Find $\lim_{x\to\infty}\dfrac{1}{x}$ and $\lim_{x\to -\infty}\dfrac{1}{x}$

As x gets larger and larger, $\dfrac{1}{x}$ close and close to 0.

So $\lim_{x\to\infty}\dfrac{1}{x}=0$.

$
\dfrac{1}{1} = 1\\
\dfrac{1}{10}= 0.1\\
\dfrac{1}{1000}=0.001\\
\dfrac{1}{1000000}=0.000001
$

If we graph this, as x gets larger, it approaches 0, as it gets smaller, it also approaches 0.

\section{Outstanding Questions?}

Is there some way to automatically insert or display the graph of an array such as $f(x) = \left\{\begin{array}{l}
x^2-1,x < 1\\
4-x, x \geq 1
\end{array} \right. $ ?

Need automated templates for the array above, and limits

The dot I wanted: $f\cdot g$

This does not place the arrow directly under the limit:


$ \lim_{x \to +\infty} \frac{3x^2 +7x^3}{x^2 +5x^4} = 3. $


Switch to amsart document class?

\section{\TeX tests}

$\lim_{x\to\infty}\dfrac{1}{x}$

 $\max_{0\le x\le 1}x(1-x)=1/4$
\end{document}


% LocalWords: LocalWords
