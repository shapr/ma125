% -*- fill-column: 110 -*-
\documentclass[12pt]{article}
\usepackage[margin=1in]{geometry}
\usepackage{hyperref}
\usepackage{amsmath}
\begin{document}
\section*{Calculus}
\section{Review}
\subsection{Factoring}
First we review basic factoring.

\begin{math}
a^2-b^2 = (a+b)(a-b)\\
a^3-b^3 = (a-b)(a^2+ab+b^2)\\
a^3+b^3 = (a+b)(a^2-ab+b^2)\\
a^2+2ab+b^2 = (a+b)^2\\
a^2-2ab+b^2 = (a-b)^2
\end{math}

\subsection{Radicals}

$\sqrt[n]{x} = x^{\frac{1}{n}}$

So what would be the result of $ \sqrt[n]{\frac{x}{y}} $ ?
\section{2013-01-14 Mon limit of a function and limit laws}
start with 2.2 and will pick up 2.1 later

consider $f(x) = \frac{x^3-x^2}{2x-2}$ near c = 1

we want to know the behavior of function f near c

\begin{flalign}
x &= f(x)\\
0.9 &= 0.405\\
0.99 &= 0.49005\\
0.9999 &= 0.499900005\\
1.0001 &= 0.500100005\\
1.01 &= 0.51005\\
1.1 &= 0.605\\
\end{flalign}
it appears that $f(x) = \frac{x^3-x^2}{2x-2}$ near c = 1 gets closer
to 0.5 as x gets closer to 1.

Notice that $f(x) = \frac{x^3-x^2}{2x-2} = \frac{x^2(x-1)}{2(x-1)} =
\frac{x^2}{2}$ if x is $\neq$ to 1
and f(x) is undefined if x = 1

(insert parabola graph of this equation)

we can see from the graph that as x gets closer to 1, f(x) gets closer
to $\frac{1}{2}$.

We say that the limit of the function $\frac{x^3-x^2}{2x-2}$ as
x approaches 1 is equal to $\frac{1}{2}$.

$\lim_{x \to 1} \frac{x^3-x^2}{2x-2} = \frac{1}{2}$ (how do I get the
lim stuff COMPLETELY under the lim word?)

\subsection{Informal definition of a limit}
suppose f(x) is defined on an open interval about c, except possibly
at c itself. If f(x) is arbitrarily close to L (as close to L as we
like) for all x sufficiently close to c, we say that f approaches the
limit L as x approaches c, and write $\lim_{x \to c}, f(x) = L $

NOTE it is important to note that $\lim_{x \to c} f(x)$
DOES NOT depend on the value of f at c.
Instead what matters is the value of f NEAR c.

Now we graph a function f 

funny triangle I don't quite understand
at zero it's 7, at 2 it's 3 at 3.5 it's roughly zero
$\lim_{x \to 2} f(x) = 3$

now we put a hole at x = 2, showing that x is not defined there.
so we got closer and closer to 3, but never quite reach it.

We draw the graph again, but we add a point 2,6 such that the y value AT 2 is definitely 6. Still doesn't
change the result of the limit because what matters is NEAR c, not AT c.

Another graph, 
$f(x) = \left\{ \begin{array}{l}
1, x < 0\\
2, x \geq 0 \end{array} \right. $
What's the limit of f(x) as it approaches 0? The limit does not exist!

but what about $f(x) = \frac{1}{x^2}$ and we want $\lim_{x \to 0} f(x)$.
We have roughly a solar chimney graph, where y asymptotically approaches 0

Is there a single finite number that our y-value is approaching as our x gets closer to zero? NO!
because this increases without bound

next graph is: $f(x) = \left\{\begin{array}{l}
0,x \leq 0\\
sin \frac{1}{x}, x > 0
\end{array} \right. $
limit does not exist because the value is oscillating

\subsection{Limit Laws}
If L,M, c and k are real numbers and 
$\lim_{x \to c} f(x) = L$ and $\lim_{x \to c} g(x) = M$, then 
$\lim_{x \to c}(f(x) + g(x)) = L+M$
that is, the sum of the limits is the limit of the sums?
$\lim_{x \to c}(f(x) - g(x)) = L-M$
that requires the difference to be the same
$\lim_{x \to c}( k \times f(x)) = k \times L$

\end{document}
% LocalWords: LocalWords
