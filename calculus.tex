% -*- fill-column: 110 -*-
\documentclass[12pt]{article}
\usepackage[margin=1in]{geometry}
\usepackage{hyperref}
\usepackage{amsmath}
\usepackage{amsthm}
\begin{document}
\section*{Calculus}
\section{Review}

\subsection{Factoring}
First we review basic factoring.

\begin{math}
a^2-b^2 = (a+b)(a-b)\\
a^3-b^3 = (a-b)(a^2+ab+b^2)\\
a^3+b^3 = (a+b)(a^2-ab+b^2)\\
a^2+2ab+b^2 = (a+b)^2\\
a^2-2ab+b^2 = (a-b)^2
\end{math}

\subsection{Radicals}

$\sqrt[n]{x} = x^{\frac{1}{n}}$

So what would be the result of $ \sqrt[n]{\frac{x}{y}} $ ?

more radicals

$x^{-\frac{1}{n}} = \dfrac{1}{x^{\frac{1}{n}}} = \dfrac{1}{\sqrt[n]{x}}$
\section{2013-01-14 Mon limit of a function and limit laws}
start with 2.2 and will pick up 2.1 later

consider $f(x) = \frac{x^3-x^2}{2x-2}$ near c = 1

we want to know the behavior of function f near c

\begin{flalign}
x &= f(x)\\
0.9 &= 0.405\\
0.99 &= 0.49005\\
0.9999 &= 0.499900005\\
1.0001 &= 0.500100005\\
1.01 &= 0.51005\\
1.1 &= 0.605\\
\end{flalign}
it appears that $f(x) = \frac{x^3-x^2}{2x-2}$ near c = 1 gets closer
to 0.5 as x gets closer to 1.

Notice that $f(x) = \frac{x^3-x^2}{2x-2} = \frac{x^2(x-1)}{2(x-1)} =
\frac{x^2}{2}$ if x is $\neq$ to 1
and f(x) is undefined if x = 1

(insert parabola graph of this equation)

we can see from the graph that as x gets closer to 1, f(x) gets closer
to $\frac{1}{2}$.

We say that the limit of the function $\frac{x^3-x^2}{2x-2}$ as
x approaches 1 is equal to $\frac{1}{2}$.

$\lim_{x \to 1} \frac{x^3-x^2}{2x-2} = \frac{1}{2}$ (how do I get the
lim stuff COMPLETELY under the lim word?)

\subsection{Informal definition of a limit}
suppose f(x) is defined on an open interval about c, except possibly
at c itself. If f(x) is arbitrarily close to L (as close to L as we
like) for all x sufficiently close to c, we say that f approaches the
limit L as x approaches c, and write $\lim_{x \to c}, f(x) = L $

NOTE it is important to note that $\lim_{x \to c} f(x)$
DOES NOT depend on the value of f at c.
Instead what matters is the value of f NEAR c.

Now we graph a function f 

funny triangle I don't quite understand
at zero it's 7, at 2 it's 3 at 3.5 it's roughly zero
$\lim_{x \to 2} f(x) = 3$

now we put a hole at x = 2, showing that x is not defined there.
so we got closer and closer to 3, but never quite reach it.

We draw the graph again, but we add a point 2,6 such that the y value AT 2 is definitely 6. Still doesn't
change the result of the limit because what matters is NEAR c, not AT c.

Another graph, 
$f(x) = \left\{ \begin{array}{l}
1, x < 0\\
2, x \geq 0 \end{array} \right. $
What's the limit of f(x) as it approaches 0? The limit does not exist!

but what about $f(x) = \frac{1}{x^2}$ and we want $\lim_{x \to 0} f(x)$.
We have roughly a solar chimney graph, where y asymptotically approaches 0

Is there a single finite number that our y-value is approaching as our x gets closer to zero? NO!
because this increases without bound

next graph is: $f(x) = \left\{\begin{array}{l}
0,x \leq 0\\
sin \frac{1}{x}, x > 0
\end{array} \right. $
limit does not exist because the value is oscillating

\subsection{Limit Laws}
If L,M, c and k are real numbers and 
$\lim_{x \to c} f(x) = L$ and $\lim_{x \to c} g(x) = M$, then 
$\lim_{x \to c}(f(x) + g(x)) = L+M$
that is, the sum of the limits is the limit of the sums?
$\lim_{x \to c}(f(x) - g(x)) = L-M$
that requires the difference to be the same
$\lim_{x \to c}( k \times f(x)) = k \times L$

\section{2013-01-15 More Limit Laws}

$\lim_{x \to c}(f(x) \times g(x)) = \lim_{x \to c}f(x) \times \lim_{x \to c}g(x) = L \times M$

$\lim_{x \to c} \frac{f(x)}{g(x)} = \lim_{x \to c} \frac{\lim_{x \to c} f(x)}{\lim_{x \to c} g(x)} \lim_{x \to
  c}g(x) \neq 0 = \frac{L}{M},M \neq 0$

$\lim_{x \to c}[f(x)]^n = [\lim_{x \to c}f(x)]^n = L^n$,n is a positive integer
the previous line expands to $\lim_{x \to c}[f(x) \times f(x) ... f(x)]$ for each n in the exponent.
and is the same as $=L \times L ... L = L^n$

$\lim_{x \to c}\sqrt[n]{f(x)} = \sqrt[n]{\lim_{x\to c}f(x)} = \sqrt[n]{L}$
If n is even, we assume $\lim_{x \to c}f(x) = L > 0$

NB. Our instructor suspects that the above law is actually $L \geq 0$ because you can take the nth root of 0.

\subsection{Basic Limits}

$\lim_{x \to c}x = C$
if we graphed this as y = x, it's just a line with slope of positive 1
as x gets closer to C, x gets closer to C. Obvious.

$\lim_{x \to c}k = k$ as x approaches k, x approaches k!

$$
\lim_{x \to 8} x = 8\\
\lim_{y \to 23} y = 23\\
\lim_{h \to 0} 7 = 7
$$

Find the following limits
$\lim_{x \to 5}2x^2 - 10\\
=\lim_{x \to 5}2x^2 - \lim_{x \to 5}10\\
=2\lim_{x \to 5}x^2 - \lim_{x \to 5}10\\
=2(\lim_{x \to 5}x)^2 - \lim_{x \to 5}10\\
=2(5)^2 - 10\\
=2(25) - 10\\
= 40$

$\lim_{x \to 4}\frac{3x+1}{x+2}\\
=\frac{\lim_{x \to 4}3x+1}{\lim_{x \to 4}x+2}\\
=\frac{3\lim_{x \to 4}x+\lim_{x \to 4}1}{\lim_{x \to 4}x+\lim_{x \to 4}2}\\
=\frac{3(4)+1}{4+2}
$

\subsection{Limits of Polynomials}
If a polynomial $P(x) = a_n x^n + a_{n-1}x^{n-1}+ ... + a_1x + a_0$, then $\lim_{x \to c} P(x)= P(c)$

Instructor points out that all the above steps ended up substituting the limit value into the polynomial.

\subsection{Limits of Rational Functions}
If P(x) and Q(x) are polynomials and $Q(c) \neq 0$, then $\lim_{x \to c}\frac{P(x)}{Q(x)} = \frac{P(c)}{Q(c)}$

now for some examples

$\lim_{x \to -1}(2x^3 - 3x + 1)\\
= (2(-1)^3 - 3(-1) + 1)\\
= (-2 + 3 + 1)\\
= 2$

Can we plug in zero for this one? Yes! Because the denominator will not end up being zero.
$\lim_{x \to 0}\frac{5x^2 + x - 1}{2x + 3}\\
=\frac{5(0)^2 + 0 - 1}{2(0) + 3}
$

This one will have a zero in the denominator.
$\lim_{x \to 2}\frac{x^2 +2x-8}{x-2}\\
=\lim_{x \to 2}\frac{(x-2)(x+4)}{x-2} = x+4 \mathrm{if}  x \neq 2$
since we are only interested in what happens when x is NEAR 2 but not equal to 2, we can use the expression
x+4 for our function when taking the limit as k approaches 2.

Now to actually work the problem:

$
\lim_{x \to 2}\frac{x^2 +2x-8}{x-2} = \lim_{x \to 2}\frac{(x-2)(x+4)}{x-2} = \lim_{x \to 2}(x+4) = 2+4 = 6
$

NB. Don't drop your limit too early. Even though your homework is online, don't just do it in your head. Tests
will be done on paper, so don't just give an answer.


$\lim_{x \to 25}\frac{\sqrt{x}-5}{x - 25}$ can't just plug in 25\\
$\lim_{x \to 25}\frac{\sqrt{x}-5}{x - 25} \times \frac{\sqrt{x + 5}}{\sqrt{x + 5}}$
$\lim_{x \to 25}\frac{x-25}{(x - 25)(\sqrt{x}+5}$
$\lim_{x \to 25}\frac{1}{\sqrt{x}+5}$

$\lim_{x \to -3}\frac{x+3}{\frac{1}{x}+\frac{1}{3}}$
$\lim_{x \to -3}\dfrac{x+3}{\frac{1}{x}+\frac{1}{3}} \times \dfrac{3x}{3x}$
$\lim_{x \to -3} \dfrac{(x+3)(3x)}{3+x}$
$\lim_{x \to -3} 3x$
$3(-3) = -9$

\subsection{Squeeze Theorem}

Suppose that $g(x) \leq f(x) \leq h(x) \forall x$ in some open interval\footnote{an open interval necessarily
  means that one or more end points are open (not included), that is would be an open circle in a number line} containing c, except possibly at $x=c$
itself.

$\lim{_{x \to c}}g(x) = \lim{_{x \to c}}h(x) = L$. Then  $\lim_{x \to c}f(x) = L$.
This is true because f is squeezed between h and g.
$\frac{1}{2}$

\section{2013-01-16 2.2 continued, squeeze theorem}

if $0 \leq f(x) \leq c$ for some real number c, then prove
function f is bounded between the number 0 and c
we want to prove that $\lim_{x \to 0} x^2f(x) = 0$

(to use the squeeze theorem, we need the limits of the two outer functions)

Proof
suppose that  $0 \leq f(x) \leq c$ for some real number c then
$0 \leq x^2f(x) \leq cx^2$ Also $\lim_{x \to 0}0 = 0$ and $\lim_{x \to 0}cx^2 = 0$. Thus by the squeeze
theorem, $\lim_{x \to 0}x^2f(x) = 0$.

Prove that $\lim_{x \to 0}x^4 \cos \frac{1}{x} = 0$.

We'll want to get to $\leq x^4\cos \frac{1}{x} \leq$, that is, we need the limits of the two outer functions.

We know that $-1 \leq \cos t \leq 1$ and $-1 \leq \cos \frac{1}{x} \leq 1 \forall x \neq 0$

We distribute $x^4$ so we get $-x^4 \leq x^4 \cos \frac{1}{x} \leq x^4$

\begin{proof}
Since $-1 \leq \cos t \leq 1$ for every real number t then $-1 \leq \cos \frac{1}{x} \leq 1$ for every $x \neq
0$. Thus $-x^4 \leq x^4 \cos \frac{1}{x} \forall x \neq 0$. Also $\lim_{x \to 0} -x^4 = 0$ and $\lim_{x \to 0}
x^4 = 0$. Hence, by the Squeeze Theorem, $\lim_{x \to 0} x^4 \cos \frac{1}{x} = 0.$
\end{proof}
Would be the same for $\sin$ instead of $\cos$, but would not be bounded for every x out there?

find

This should help you out with some of the homework.
$\lim_{x \to \frac{\pi}{2}} \sin x = sin \frac{\pi}{2} = 1$

\subsection{2.3 The precise definition of a limit}
If we can make f(x) arbitrarily close to L as long as we can make f(x) close to c. What?

Now we get a more formal definition of close enough.

We have agreed that $\lim_{x \to c}f(x) = L$ if we can cause f(x) to be arbitrarily close to L (as close as we
like) by taking x sufficiently close (enough) to c.
So there is a positive number $\delta$ such that x is in $(c-\delta, c+\delta)$
$c- \delta < x < c + \delta$
$-\delta < x -c < \delta$
$|x - c| < \delta$ 

The above means x is close to c.


Let's say that f(x) is close to L if f(x) lies within an open interval centered at L. So there is a positive
number $\epsilon$ such that f(x) is in $(L-\epsilon{},L+\epsilon{})$
$
L-\epsilon < f(x) < L + \epsilon
$
$-\epsilon < (x) -L < \epsilon$
$|f(x)-L| < \epsilon$

The above means f(x) is close to L.

We will now use these together.

\subsection{Formal Definition of a Limit}
Let f be a function defined on some open interval\footnote{a closed interval means that the range of the
  function is \underline{continuous} over the real numbers.} containing c, except possibly at c itself.

Then the$\lim_{x \to c} f(x) = L$ if for every $\epsilon > 0 \exists \delta > 0$ such that if $0 < |x-c| <
\delta{}$\footnote{This is where x is sufficiently close to c, note that x cannot equal c, thus we have the
  greater than zero constraint} then we will have our $|f(x)-L| < \epsilon $\footnote{f(x) is arbitrarily close to L}.

Epsilon is how far away from the limit you are, delta is how far away from the ...


$\lim_{x \to c}f(x) = L$ we have a sine graph of f, where we examine the maximum vertical interval we can have
that will satisfy $(c- \delta,c + \delta)$ 

Prove $\lim_{x \to 3}(2x +4) = 10$

\begin{proof}
Given $\epsilon > 0$. We seek a $\delta > 0$ such that if $0 < |x-c| < \delta$ then $|f(x)-L|< \epsilon$.

we substitute 3 for c and 10 for L. (why?)

if $0 < |x-3| < \delta$ then $|(2x-4)-10|< \epsilon$ 
if $0 < |x-3| < \delta = ?$ then $|(2x+4)-10| = |2x-6| = |2(x-3)| = 2|x-3| $
we know that $ 2|x-3| < 2 \delta = 2(?) = 2(\frac{\epsilon}{2} = \epsilon)$ what?
\end{proof}
\section{Outstanding Questions?}

\end{document}
% LocalWords: LocalWords
