% -*- fill-column: 110 -*-
\documentclass[12pt]{article}
\usepackage[margin=1in]{geometry}
\usepackage{hyperref}
\usepackage{amsmath} % math
\usepackage{amsthm} % theorem

\usepackage{fancyhdr}
\pagestyle{fancy}
\lhead{\footnotesize 2.5 Continuity Homework}
\rhead{\footnotesize January 31 2013 -- MA125 -- Shae Erisson}

\begin{document}
\section*{2.5 Continuity Homework}

Prove that the function $f(x) = x^3 - 4x +1$ has at least one zero in
the interval $[1,2]$

\begin{proof}
Let $f(x) = x^3 - 4x +1$. Since $f$ is a polynomial function, then it is
continuous for all real numbers. Thus $f$ is continuous on $[1,2]$.
Also, $f(1) = -2$ and $f(2) = 1$. So $f(1) < 0 < f(2)$. Hence by the intermediate value
theorem, $f(c) = 0$ for some $c$ in $[1,2]$.
\end{proof}

Prove that the equation $\sin x - x = -2$ has at least one solution.

\begin{proof}
  Let $f(x) = \sin x - x + 2$. Since $\sin$ is continuous on the real numbers, then this entire function is
  continuous for all real numbers. Thus $f$ is continuous on the closed interval $[2,3]$.  Also $f(2) = 0.9$
  and $f(3) = -0.8$ were discovered through empirical analysis. So $f(2) < 0 < f(3)$. Hence by the
  intermediate value theorem, $f(c) = 0$ for some $c$ in $[2,3]$.
\end{proof}

Prove that the equation $\sqrt{x+5}=\dfrac{1}{x+3}$ has at least one solution.

\begin{proof}
Let $f(x) = \sqrt{x+5}$ and $g(x) = \dfrac{1}{x+3}$. The function $f$ is continuous as long as the value under
the radical is greater than or equal to zero. Thus the smallest input that gives a real number output is $-5$,
and the largest is $\infty$. Thus $f(x)$ is continuous over the interval $[-5,\infty)$.
The function $g(x)$ is discontinuous only at the input of $-3$. Thus the function $g(x)$ is continuous in the
interval $(-\infty,-3)\cup(-3,\infty)$. The intersection of these two functions $[-5,-3)\cap(-3,\infty)$
describes the continuity for the entire function.

Thus we can pick two numbers where the function is continuous, such as $-2$ and $-1$, and by the intermediate
value theorem, there is an $f(c)$ for some $c$ in $[-2,-1]$.
\end{proof}

\end{document}
%%% Local Variables: 
%%% mode: latex
%%% TeX-master: t
%%% End: 
