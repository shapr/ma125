% -*- fill-column: 110 -*-
\documentclass[12pt]{article}
\usepackage[margin=1in]{geometry}
\usepackage{hyperref}
\usepackage{amsmath}
\usepackage{amsthm}

\usepackage{fancyhdr}
\pagestyle{fancy}
\lhead{\footnotesize 2.5 Continuity Homework}
\rhead{\footnotesize January 29 2013 -- MA125 -- Shae Erisson}

\begin{document}
\section*{2.5 Continuity Homework}

Prove that the function $f(x) = x^3 - 4x +1$ has at least one zero in
the interval $[1,2]$

\begin{proof}
Let $f(x) = x^3 - 4x +1$. Since $f$ is a polynomial function, then it is
continuous for all real numbers. Thus $f$ is continuous on $[1,2]$.
Also, $f(1) = -2$ and $f(2) = 1$. So $f(1) < 0 < f(2)$. Hence by the intermediate value
theorem, $f(c) = 0$ for some $c$ in $[1,2]$.
\end{proof}

Prove that the equation $\sin x - x = -2$ has at least one solution.

\begin{proof}
Let $f(x) = \sin x - x + 2$. Since $f$ is a polynomial function, then it
is continuous for all real numbers. Thus $f$ is continuous on $[2,3]$.
Also $f(2) = 0.9$ and $f(3) = -0.8$. So $f(2) < 0 < f(3)$. Hence by
the intermediate value theorem, $f(c) = 0$ for some $c$ in $[2,3]$.
\end{proof}

Prove that the equation $\sqrt{x+5}=\dfrac{1}{x+3}$ has at least one
solution.

\begin{proof}
Let $f(x) = \sqrt{x+5} - \dfrac{1}{x+3}$. Since $f$ is a polynomial function, then it
is continuous for all real numbers. Thus $f$ is continuous on $[-1,1]$.
Through empirical exploration, no input values were found that
produced negative numbers, thus no proof was found that this function
has at least one solution. Thus its is proven that I need to show up at
your office hours.
\end{proof}

\end{document}
%%% Local Variables: 
%%% mode: latex
%%% TeX-master: t
%%% End: 
